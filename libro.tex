\documentclass[11pt,openany]{book}

% ============================================
% PACKAGES
% ============================================
\usepackage[utf8]{inputenc}
\usepackage[spanish]{babel}
\usepackage[T1]{fontenc}
\usepackage{geometry}
\usepackage{setspace}
\usepackage{titlesec}
\usepackage{fancyhdr}
\usepackage{graphicx}
\usepackage{xcolor}
\usepackage{parskip}
\usepackage{microtype}
\usepackage{listings}
\usepackage{hyperref}
\usepackage{enumitem}
\usepackage{tcolorbox}

% ============================================
% PAGE GEOMETRY (Trade paperback size)
% ============================================
\geometry{
    paperwidth=6in,
    paperheight=9in,
    top=0.75in,
    bottom=0.75in,
    inner=0.75in,
    outer=0.5in
}

% ============================================
% TYPOGRAPHY
% ============================================
\usepackage{charter} % Professional serif font
\usepackage[scaled=0.95]{helvet} % Sans-serif for headings
\renewcommand{\familydefault}{\rmdefault}

% Line spacing
\setstretch{1.25}

% Paragraph settings
\setlength{\parindent}{0pt}
\setlength{\parskip}{0.5em}

% Microtype for better text
\microtypesetup{
    activate={true,nocompatibility},
    final,
    tracking=true,
    factor=1100,
    stretch=10,
    shrink=10
}

% ============================================
% COLORS
% ============================================
\definecolor{chaptercolor}{RGB}{41,128,185}  % Professional blue
\definecolor{sectioncolor}{RGB}{52,73,94}    % Dark gray
\definecolor{codebackground}{RGB}{245,245,245}
\definecolor{linkcolor}{RGB}{41,128,185}

% ============================================
% CHAPTER & SECTION FORMATTING
% ============================================
\titleformat{\chapter}[display]
{\normalfont\huge\bfseries\sffamily\color{chaptercolor}}
{\chaptertitlename\ \thechapter}
{20pt}
{\Huge}

\titleformat{\section}
{\normalfont\Large\bfseries\sffamily\color{sectioncolor}}
{\thesection}
{1em}
{}

\titleformat{\subsection}
{\normalfont\large\bfseries\sffamily\color{sectioncolor}}
{\thesubsection}
{1em}
{}

% ============================================
% HEADERS & FOOTERS
% ============================================
\pagestyle{fancy}
\fancyhf{}
\fancyhead[LE,RO]{\thepage}
\fancyhead[RE]{\textit{\leftmark}}
\fancyhead[LO]{\textit{\rightmark}}
\renewcommand{\headrulewidth}{0.4pt}
\renewcommand{\footrulewidth}{0pt}

% Plain style for chapter pages
\fancypagestyle{plain}{
    \fancyhf{}
    \fancyfoot[C]{\thepage}
    \renewcommand{\headrulewidth}{0pt}
}

% ============================================
% CODE LISTINGS
% ============================================
\lstset{
    basicstyle=\small\ttfamily,
    backgroundcolor=\color{codebackground},
    frame=single,
    rulecolor=\color{sectioncolor},
    breaklines=true,
    breakatwhitespace=true,
    numbers=none,
    xleftmargin=0.5cm,
    xrightmargin=0.5cm,
    aboveskip=1em,
    belowskip=1em
}

% ============================================
% HYPERLINKS
% ============================================
\hypersetup{
    colorlinks=true,
    linkcolor=linkcolor,
    urlcolor=linkcolor,
    citecolor=linkcolor,
    bookmarksopen=true,
    pdfstartview=FitH
}

% ============================================
% CUSTOM BOXES FOR TIPS/NOTES
% ============================================
\newtcolorbox{tipbox}[1][]{
    colback=codebackground,
    colframe=chaptercolor,
    fonttitle=\bfseries\sffamily,
    title={#1},
    arc=2mm,
    boxrule=1pt
}

% ============================================
% LIST FORMATTING
% ============================================
\setlist[itemize]{leftmargin=*,noitemsep,topsep=0.5em}
\setlist[enumerate]{leftmargin=*,noitemsep,topsep=0.5em}

% ============================================
% DOCUMENT METADATA
% ============================================
\title{Agilmente para Developers}
\author{[Tu Nombre]}
\date{\today}

% ============================================
% BEGIN DOCUMENT
% ============================================
\begin{document}

% ============================================
% FRONT MATTER
% ============================================
\frontmatter

% Title Page
\begin{titlepage}
    \centering
    \vspace*{2cm}

    {\Huge\bfseries\sffamily\color{chaptercolor} AGILMENTE\\PARA DEVELOPERS\par}

    \vspace{1.5cm}

    {\Large\sffamily Productividad, Flow y Código Brillante\\Sin Quemarte en el Intento\par}

    \vspace{3cm}

    {\Large\itshape [Tu Nombre]\par}

    \vfill

    {\large \today\par}
\end{titlepage}

% Copyright Page
\newpage
\thispagestyle{empty}
\vspace*{\fill}
\begin{center}
    \textbf{Agilmente para Developers}\\
    Productividad, Flow y Código Brillante Sin Quemarte en el Intento\\[1em]

    Copyright \copyright\ 2025 [Tu Nombre]\\[1em]

    Todos los derechos reservados.\\[1em]

    Ninguna parte de este libro puede ser reproducida, almacenada en un sistema\\
    de recuperación o transmitida de ninguna forma o por ningún medio,\\
    electrónico, mecánico, fotocopia, grabación u otro,\\
    sin el permiso previo por escrito del autor.\\[2em]

    Primera Edición, 2025\\[1em]

    ISBN: [Tu ISBN]\\[2em]

    \textit{Generado con \LaTeX}
\end{center}
\vspace*{\fill}

% Dedication (optional)
\newpage
\thispagestyle{empty}
\vspace*{\fill}
\begin{center}
    \textit{Para todos los developers que alguna vez pensaron\\
    que trabajar más horas era la solución.\\[1em]
    No lo es.\\[1em]
    Este libro es tu permiso para trabajar diferente.}
\end{center}
\vspace*{\fill}

% Table of Contents
\tableofcontents

% ============================================
% MAIN MATTER
% ============================================
\mainmatter

% ============================================
% INTRODUCCIÓN
% ============================================
\chapter*{Introducción: El Día Que Todo Cambió}
\addcontentsline{toc}{chapter}{Introducción: El Día Que Todo Cambió}

Estaba sentado frente a mi laptop a las 11 de la noche, por quinto día consecutivo. Mis ojos ardían. Mi espalda dolía. Había estado ``trabajando'' 12 horas, pero cuando revisé qué había logrado, la respuesta era deprimente: casi nada.

[...]

% ============================================
% CHAPTER 1
% ============================================
\chapter{El Bug en Tu Cerebro}

Era martes. Las 3 de la tarde. Llevaba seis horas sentado frente a la pantalla y no había avanzado nada. NADA. [...]

\section{El descubrimiento que cambió todo}

Dos años después de esa tarde horrible, leí un paper que me voló la cabeza. [...]

\section{Tu cerebro no es una computadora}

Durante años me comparé con mi laptop. [...]

\section{Las cinco tareas simultáneas del programador}

Volvamos al estudio de Siegmund. [...]

\subsection{1. Comprender sintaxis}
\subsection{2. Mantener el contexto en memoria de trabajo}
\subsection{3. Construir modelos mentales}
\subsection{4. Resolver problemas}
\subsection{5. Planificar y ejecutar}

\section{Tres cosas que puedes hacer hoy}

\begin{tipbox}[Acción Inmediata]
\textbf{1. Mide tu energía cognitiva}\\
Durante una semana, cada dos horas pregúntate: ``Del 1 al 10, ¿cuánta energía mental tengo?''

\textbf{2. Respeta el límite de 90 minutos}\\
Tu cerebro trabaja en ciclos ultradianos de 90-120 minutos.

\textbf{3. Externaliza tu memoria de trabajo}\\
Antes de abrir el IDE, tomá un papel y escribí: ``¿Qué voy a hacer?''
\end{tipbox}

% ============================================
% CHAPTER 2
% ============================================
\chapter{El Mito del Multitasking}

[Contenido del capítulo...]

% ============================================
% CHAPTER 3
% ============================================
\chapter{Tu Cerebro en Flow}

[Contenido del capítulo...]

% ============================================
% CHAPTER 4
% ============================================
\chapter{El Poder del Descanso}

[Contenido del capítulo...]

% ============================================
% CHAPTER 5
% ============================================
\chapter{IA: Tu Copiloto, No Tu Piloto}

[Contenido del capítulo...]

% ============================================
% CHAPTER 6
% ============================================
\chapter{Programar como un Estoico}

[Contenido del capítulo...]

% ============================================
% CHAPTER 7
% ============================================
\chapter{El Código Simple Gana}

[Contenido del capítulo...]

% ============================================
% CHAPTER 8
% ============================================
\chapter{Empieza Hoy}

[Contenido del capítulo...]

% ============================================
% EPILOGUE
% ============================================
\chapter*{Epílogo: El Desarrollador que Fuiste, el Desarrollador que Eres, el Desarrollador que Serás}
\addcontentsline{toc}{chapter}{Epílogo}

Cuando empezaste a programar, probablemente creías que ser buen developer era sobre conocer más sintaxis, más frameworks, más patrones. [...]

% ============================================
% BACK MATTER
% ============================================
\backmatter

% ============================================
% BIBLIOGRAPHY (optional)
% ============================================
\chapter*{Referencias y Estudios Citados}
\addcontentsline{toc}{chapter}{Referencias y Estudios Citados}

\begin{itemize}
    \item Siegmund, J., et al. (2014). ``Understanding Understanding Source Code with Functional Magnetic Resonance Imaging.'' Proceedings of ICSE.

    \item Nass, C., et al. (2009). ``Cognitive Control in Media Multitaskers.'' Stanford University, PNAS.

    \item Mark, G., et al. (2008). ``The Cost of Interrupted Work: More Speed and Stress.'' CHI 2008, University of California, Irvine.

    \item Leroy, S. (2009). ``Why is it so hard to do my work? The challenge of attention residue when switching between work tasks.'' Organizational Behavior and Human Decision Processes.

    \item Csikszentmihalyi, M. (1990). ``Flow: The Psychology of Optimal Experience.'' Harper \& Row.

    \item McKinsey \& Company (2014). ``Increasing the 'meaning quotient' of work.'' McKinsey Quarterly.

    \item Microsoft Japan (2019). ``Work-Life Choice Challenge Summer 2019.'' Microsoft Corporation.

    \item Walker, M. (2017). ``Why We Sleep: Unlocking the Power of Sleep and Dreams.'' Scribner.

    \item GitHub (2022). ``Research: Quantifying GitHub Copilot's impact on developer productivity and happiness.'' GitHub Blog.

    \item Peng, S., et al. (2023). ``Do Users Write More Insecure Code with AI Assistants?'' Stanford University.

    \item Brooks, F. (1987). ``No Silver Bullet—Essence and Accident in Software Engineering.'' IEEE Computer.
\end{itemize}

% ============================================
% ABOUT THE AUTHOR (optional)
% ============================================
\chapter*{Sobre el Autor}
\addcontentsline{toc}{chapter}{Sobre el Autor}

[Tu biografía aquí]

\end{document}
